% Document class
\documentclass{article}

% For figures
\usepackage{graphicx} % more modern
\usepackage{subfigure} 
% For citations
\usepackage{natbib}

% For algorithms
\usepackage{algorithm}
%\usepackage{algorithmic}

% As of 2011, we use the hyperref package to produce hyperlinks in the
% resulting PDF.  If this breaks your system, please commend out the
% following usepackage line and replace \usepackage{cs4437cs9637} with
% \usepackage[nohyperref]{cs4437cs9637} above.
\usepackage{hyperref}

% Packages hyperref and algorithmic misbehave sometimes.  We can fix
% this with the following command.
\newcommand{\theHalgorithm}{\arabic{algorithm}}

% Employ the following version of the ``usepackage'' statement for
% submitting the draft version of the paper for review.  This will set
% the note in the first column to ``Under review.  Do not distribute.''
\usepackage{cs4437cs9637} 

\begin{document}

% The \cstitle you define below is probably too long as a header.
% Therefore, a short form for the running title is supplied here:
\cstitlerunning{Project Proposal}

\twocolumn[
\cstitle{Evaluating Patterns in Critically Acclaimed Music}

% It is OKAY to include author information
\csauthor{Paul Bartlett (\normalsize\emph{\# 250753008})}{\href{mailto:pbartle7@uwo.ca}{\nolinkurl{pbartle7@uwo.ca}}}
\csaddress{The University Of Western Ontario}

% You may provide any keywords that you 
% find helpful for describing your paper; these are used to populate 
% the "keywords" metadata in the PDF but will not be shown in the document
\cskeywords{}

\vskip 0.3in
]

\begin{abstract} 
{\bf } The purpose of this analysis is to identify relationships between
musical genre of critically acclaimed albums and time. The dataset used
for this analysis contains over 18,000 reviews from Pitchfork from
January 5th, 1999 to January 8th, 2017. It contains important data
including release year, artist name, genre, and a score ranging from
0.0-10.0. The findings may be useful for determining what the most
successful genre of critically acclaimed music is for each of the last
18 years. \end{abstract} 


\section{Description of Applied
Problem}\label{description-of-applied-problem}

The trends of popular music can easily be attained through the various
Billboard charts that have existed since 1955. A group of scientists
from the University of London analysed around 17,000 songs that charted
on the U.S. Billboard Hot 100 over the last 50 years and created a
visualization of the popularity of musical genres over time
\citep{Billboard100}. The problem with getting data from these charts is
that popular music generally isn't critically acclaimed, and is
therefore not as interesting as data from sources that evaluate music
more objectively. Using a dataset that includes over 18,000 reviews from
Pitchfork, I will be going through the data to find how critically
acclaimed music has changed over time. In addition, I will also be
looking at which release from an artist is the most well received. A
general pattern I've seen when listening to several albums from an
artist is that the first 2-3 albums tend to be the best from their
discography. I would like to use the data to confirm or deny this
assumption. and the charts don't tell us about the musical trends of the
highest rated artists and albums.
\url{http://www.latimes.com/visuals/graphics/la-sci-g-music-evolution-20150505-htmlstory.html}
\url{https://musicmap.info/}

\section{Description of Available
Data}\label{description-of-available-data}

The data set is available on kaggle for
\href{https://www.kaggle.com/nolanbconaway/pitchfork-data}{Pitchfork
Reviews from January 5th, 1999 to January 8th, 2017}.

\bibliography{Evaluating Patterns in Critically Acclaimed Music}
\bibliographystyle{plainnat}

\end{document} 
