% Document class
\documentclass{article}

% For figures
\usepackage{graphicx} % more modern
\usepackage{subfigure} 
% For citations
\usepackage{natbib}

% For algorithms
\usepackage{algorithm}
%\usepackage{algorithmic}

% As of 2011, we use the hyperref package to produce hyperlinks in the
% resulting PDF.  If this breaks your system, please commend out the
% following usepackage line and replace \usepackage{cs4437cs9637} with
% \usepackage[nohyperref]{cs4437cs9637} above.
\usepackage{hyperref}

% Packages hyperref and algorithmic misbehave sometimes.  We can fix
% this with the following command.
\newcommand{\theHalgorithm}{\arabic{algorithm}}

% Employ the following version of the ``usepackage'' statement for
% submitting the draft version of the paper for review.  This will set
% the note in the first column to ``Under review.  Do not distribute.''
\usepackage{cs4437cs9637} 

\begin{document}

% The \cstitle you define below is probably too long as a header.
% Therefore, a short form for the running title is supplied here:
\cstitlerunning{Project Proposal}

\twocolumn[
\cstitle{Evaluating Patterns in Critically Acclaimed Music}

% It is OKAY to include author information
\csauthor{Paul Bartlett (\normalsize\emph{\# 250753008})}{\href{mailto:pbartle7@uwo.ca}{\nolinkurl{pbartle7@uwo.ca}}}
\csaddress{The University Of Western Ontario}

% You may provide any keywords that you 
% find helpful for describing your paper; these are used to populate 
% the "keywords" metadata in the PDF but will not be shown in the document
\cskeywords{}

\vskip 0.3in
]

\begin{abstract} 
{\bf } The purpose of this analysis is to identify relationships between
musical genre of critically acclaimed albums and time. The dataset used
for this analysis contains over 18,000 reviews from Pitchfork from
January 5th, 1999 to January 8th, 2017. It contains important data
including release year, artist name, genre, and a score ranging from
0.0-10.0. The findings may be useful for determining what the most
successful genre of critically acclaimed music is for each of the last
18 years. \end{abstract} 


\section{Description of Applied
Problem}\label{description-of-applied-problem}

\subsection{Existing solutions to similar
problems}\label{existing-solutions-to-similar-problems}

The trends of popular music can easily be attained through the various
Billboard charts that have existed since 1955. A group of scientists
from the University of London analysed around 17,000 songs that charted
on the U.S. Billboard Hot 100 over the last 50 years and created a
visualization of the popularity of musical genres over time
\citep{Billboard100}. The problem with getting data from these charts is
that popular music generally isn't critically acclaimed, and is
therefore not as interesting as data from sources that evaluate music
more objectively. Another source that uses visualization of this problem
well is musicmap \citep{musicmap}. The website contains information
about hundreds of genres of music and their history. It provides a great
overview of all the popular strands of music, but doesn't go into too
much depth about specific artists or albums. It does provide a good
overview of all genres regardless of popularity, but I'm more interested
in evaluating the history of the best albums created by artists.

\subsection{Pitchfork solution}\label{pitchfork-solution}

Using a dataset that includes over 18,000 reviews from Pitchfork, I will
be going through the data to find how critically acclaimed music has
changed over time. In addition, I will also be looking at which release
from an artist is the most well received. A general pattern I've seen
when listening to several albums from an artist is that the first 2-3
albums tend to be the best from their discography. I would like to use
the data to confirm or deny this assumption.

\section{Description of Available
Data}\label{description-of-available-data}

\subsection{Pitchfork}\label{pitchfork}

The dataset that I will be using is taken from Pitchfork. Pitchfork is
an online magazine that focuses on reviewing both popular and
independant music. The data set for Pitchfork Reviews from January 5th,
1999 to January 8th, 2017 is available on kaggle \citep{kaggle}. There
are 18,393 reviews that include important data including release year,
artist name, genre, and a score ranging from 0.0-10.0. Considering that
Pitchfork is one of the longest running online review sites, it makes it
a primary choice for useful data.

\subsection{Best New Music}\label{best-new-music}

Pitchfork features a section called ``Best New Music'' for highlighting
recent releases that the staff found to stand out in a positive way.
These albums generally have a minimum score of 8.0 and are another
useful way for tracking the best music on the site. The dataset includes
an identifier for ``Best New Music'', and would be a useful way to sort
through the data.

\section{Plan for Analysis and
Visualization}\label{plan-for-analysis-and-visualization}

\subsection{Analysis}\label{analysis}

To analyse this data properly, we must use a method to take out only the
best reviews from the dataset. Fortunately, Pitchfork has a system to
distinguish the best albums called ``Best New Music''. Unfortunately,
this feature launched in 2003 so using it would leave out all the music
before it was launched. By looking at the data, we will be able to find
the typical rating for an album that gets the ``Best New Music'' tag and
use that rating to take all albums from the data set that are higher
than the threshold. From this, we should be able to classify each album
that meets the requirement by year and genre so that it can be used for
visualization. To analyse what release number is considered to be the
best, we will have to query all artists that have multiple releases on
the site. A similar analysis and visualization was done on kaggle by the
author of the data \citep{kaggleFirst}.

\subsection{Visualization}\label{visualization}

For visualization, I would like to do something similar to what was done
for visualizing the U.S. Billboard Hot 100 over the last 50 years
\citep{BillboardFigure}. The chart features spindles for each genre that
run vertically with the width proportional to the frequency of the
style. The y-axis contains the year so the viewer can easily compare
between each year to see what genre is the most popular or the least
popular. Since I do not have much experience with visualization, it is
possible that doing something similar will be too difficult to achieve.
A simple way to visualize this in a similar way would be to use a line
graph, with each line representing a genre, the x-axis covering each
year, and the y-axis covering the frequency.

\bibliography{bibliography.bib}
\bibliographystyle{plainnat}

\end{document} 
